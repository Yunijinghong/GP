%!TEX program = xelatex
\documentclass[dvipsnames, svgnames,a4paper,11pt]{article}
\input{Settings} 
\usepackage{lipsum}
\usepackage{adjustbox}
\usepackage{lipsum}
\usepackage{enumitem}
\usepackage{tabularray}
%\usepackage{mathrsfs} % 字体
%\captionsetup[figure]{name=Figure} % 图片形式
%\captionsetup[table]{name=Table} % 表格形式
\begin{document}
	
	% 实验报告封面	
	% 顶栏
	\begin{table}
		\renewcommand\arraystretch{1.7}
		\begin{tabularx}{\textwidth}{
				|X|X|X|X
				|X|X|X|X|}
			\hline
			\multicolumn{2}{|c|}{预习报告}&\multicolumn{2}{|c|}{实验记录}&\multicolumn{2}{|c|}{分析讨论}&\multicolumn{2}{|c|}{总成绩}\\
			\hline
			\LARGE25 & & \LARGE25 & & \LARGE30 & & \LARGE80 & \\
			\hline
		\end{tabularx}
	\end{table}
	% ---
	
	% 信息栏
	\begin{table}
		\renewcommand\arraystretch{1.7}
		\begin{tabularx}{\textwidth}{|X|X|X|X|}
			\hline
			年级、专业: & 2022级 物理学 &组号: &实验组1 \\
			\hline
			姓名: &   黄罗琳 & 学号: &  22344001 \\
			\hline
			实验时间: & 2024/4/18 & 教师签名: & \\
			\hline
		\end{tabularx}
	\end{table}
	% ---
	
	% 大标题
	\begin{center}
		\LARGE CA3 \quad 原子的发射和吸收光谱观测分析实验
	\end{center}
	
	\textbf{【实验报告注意事项】}
	\begin{enumerate}
		
		\item 实验报告由三部分组成:
	\begin{enumerate}[label=\textup{(\arabic*)}]
		\item 预习报告:课前认真研读\underline{\textbf{实验讲义}},实验所需的仪器设备、用具及其使用、完成课前预习思考题;了解实验需要测量的物理量,并根据要求提前准备实验记录表格(可以参考实验报告模板,可以打印)。\textcolor{red}{\textbf{(25分)}}
		
	    \item 实验记录:认真、客观记录实验条件、实验过程中的现象以及数据。实验记录请用珠笔或者钢笔书写并签名(\textcolor{red}{\textbf{用铅笔记录的被认为无效}})。\textcolor{red}{\textbf{保持原始记录,包括写错删除部分,如因误记需要修改记录,必须按规范修改。}}(不得输入电脑打印,但可扫描手记后打印扫描件);离开前请实验教师检查记录并签名。\textcolor{red}{\textbf{(30分)}}
	    
	    \item 数据处理及分析讨论:处理实验原始数据(学习仪器使用类型的实验除外),对数据的可靠性和合理性进行分析;按规范呈现数据和结果(图、表),包括数据、图表按顺序编号及其引用;分析物理现象(含回答实验思考题,写出问题思考过程,必要时按规范引用数据);最后得出结论。\textcolor{red}{\textbf{(25分)}}
	    
	\end{enumerate}
	\textbf{实验报告就是将预习报告、实验记录、和数据处理与分析合起来,加上本页封面。}
	\item 每次完成实验后的一周内交\textbf{实验报告}(特殊情况不能超过两周)。
	\item 注意事项:
		\begin{enumerate}[label=\textup{(\arabic*)}]
			\item 实验中\textcolor{red}{\textbf{光纤不能过度弯折}};
			\item 信号强度不能过饱和值;
			\item 光源长时间通电后会\textcolor{red}{\textbf{发热}},小心烫手,切换光源时务必注意(可等断电冷却后再碰);
			\item \textcolor{red}{\textbf{请提前了解光纤光谱仪的基本工作原理与关键参数等。}}
		\end{enumerate}
\end{enumerate}

	
	
	\clearpage
	\tableofcontents
	\clearpage
	
	\setcounter{section}{0}
	\section{CA3 \quad 原子的发射和吸收光谱观测分析实验 \quad\heiti 预习报告}
		
	\subsection{实验目的}
		\begin{enumerate}
			\item 原子发射光谱的观测:
				\begin{enumerate}
					\item 学习光纤光谱仪的使用;
					\item 观测钠原子光谱,了解碱金属原子光谱的一般规律;
					\item 观测汞原子光谱,了解中外层电子与原子核相互作用;
					\item 观测多种光源的发射光谱,了解线光谱与连续谱的异同。
				\end{enumerate}
			\item  原子吸收光谱的观测:
				\begin{enumerate}
					\item 调配不同浓度的高锰酸钾水溶液;
					\item 测量高锰酸钾水溶液的紫外-可见吸收光谱,找出吸收峰;
					\item 测量不同浓度高锰酸钾水溶液的紫外-可见吸收光谱,验证比尔定律;
					\item 测量不同片数玻璃基板的透过光谱,验证朗伯定律。
				\end{enumerate}
		\end{enumerate}
		
	
	\subsection{仪器用具}
		\begin{table}[htbp]
			\centering
			\renewcommand\arraystretch{1.6}
			% \setlength{\tabcolsep}{10mm}
			\begin{tabular}{p{0.05\textwidth}|p{0.20\textwidth}|p{0.05\textwidth}|p{0.5\textwidth}}
			\hline
			编号& 仪器用具名称 & 数量 &  主要参数(型号,测量范围,测量精度等) \\
			\hline
			1 & 多种光源 	& 若干	& {\footnotesize 低压汞灯、低压钠灯、氢氘灯、 溴钨灯、多种颜色的发光二极管} \\
		
			2 & 滤光片 	& 2 	& 白片、红片 \\
			
			3 & 测控计算机 & 1 &  \\
			
			4 & 光谱观测和分析仪器 & 1 & 光纤光谱仪\\
			
			5 & 高锰酸钾水溶液 & -- &  \\
			
			6 & 玻璃基板 & 1  &  \\
			
			7 & 比色皿 & 1 & \\
			
			\hline
			\end{tabular}
		\end{table}
	
	\subsection{原理概述}
	
			\begin{enumerate}
			\item 原子发射光谱的观测:
				\begin{enumerate}
					\item \textbf{碱金属原子光谱}:\\
						碱金属和氢原子一样,核外只有一个价电子,但在碱金属原子中除了一个价电子外,还有封闭在内的壳层电子,这些内封闭的电子和原子核统称为原子实。当价电子贯穿原子实时,会产生异于氢原子光谱的一系列特点。碱金属原子光谱线公式为:
						\[
						\widetilde{v}=R(\frac{1}{n_2^{*2}}-\frac{1}{n_1^{*2}})=\frac{R}{{(n'-{\mu'}_{l'})}^2}-\frac{R}{{(n-\mu_l)}^2}
						\]
						其中,$\widetilde{v}$为光谱线的波数;\\ $R$ 为里德堡常数;\\$n'$与$n$分别为始态和终态的主量子数;\\$n_2^{*}$与$n_1^{*}$分别为始态和终态的有效量子数;\\$l′$与$l$分别为该量子数决定之能级的轨道量子数;\\$𝜇{\mu'}_{l'}$与$\mu_l$分别为始态和终态的量子缺(也称量子改正数,量子亏损)。
						
						以钠原子为例来说,它的光谱分四个线系:主线系、锐线系、漫线系、基线系。对于某一线系谱线的波数公式可写为:
						\[\tilde{\nu} = A_{n'l'}-\dfrac{R}{(n-\mu_{l})^2}\]
						
						从钠原子光谱中,可以看出各个线系的一些明显特征,这些特征也为其它碱金属原子光谱所具有。
						\textbf{各线系的共同特点是}:
							\begin{enumerate}
								\item 随着波长减小,同一线系内相邻谱线的波数差逐渐减小,最终趋于一个极限,这是由于能级间距随能级升高而变小的结果。
								\item 同一线系内随着波长减小,谱线的强度逐渐减小,这是因为激发原子到高能级的能量随之增加,导致激发的难度增大。
							\end{enumerate}
						\textbf{各线系的区别}:
						

						\begin{enumerate}
							\item \textbf{光谱区域分布}:
							\begin{itemize}
								\item 主线系的谱线大部分位于紫外区域,只有钠的双黄线($\widetilde{v}=3p\rightarrow 3s$)在可见光区域,波长分别为589.0nm和589.6nm。由于主线系的下能级是基态($3s_{1/2}$能级),因此当具有连续谱的光谱通过钠原子蒸汽经过分光后,在连续光谱的背景上将出现钠原子主线系的吸收光谱,称为共振线。锐线系和漫线系的谱线大部分位于可见光区域,
							\end{itemize}
							\item \textbf{能级简并性}:
							\begin{itemize}
								\item 在碱金属原子中,$s$能级是无简并的,而$p$、$d$、$f$能级由于电子自旋与轨道运动作用引起谱项分裂,因此是双重简并的。这种双重分裂随能级增高而逐渐减小。根据选择定则,主线系和锐线系的谱线是双线的,其波数差随着能级的增加而变小。而漫线系和基线系的谱线则呈现复双重线的形态。
							\end{itemize}
							\item \textbf{谱线外观}:
							\begin{itemize}
								\item 从谱线的外观来看,主线系的谱线强度较大,锐线系的谱线轮廓清晰,而漫线系的谱线则显得比较弥漫,一般复双重线连成一片。
							\end{itemize}
						\end{enumerate}
						
				
				\end{enumerate}
				
				
				
			\item \textbf{原子吸收光谱的观测}:
				\begin{enumerate}
					\item \textbf{光的吸收}:
						
						在吸收过程中,物质的原子或分子吸收了入射的辐射能,从基态跃迁至高能级的激发态,吸收的能量与电磁辐射的频率成正比,符合普朗克公式:
						\[ E = h\nu \]
						光的吸收是指光波穿过介质后光强减弱的现象。除了真空外,几乎所有介质对电磁波都不完全透明,都会发生吸收。根据吸收特性,吸收可分为一般吸收和选择吸收。一般吸收是指在一定波长范围内,物质对光的吸收不随波长变化;选择吸收则是指吸收随波长变化的现象。物质分子的能级结构决定了其吸收电磁辐射的能力,能级间的能量差越大,吸收越小,形成了一般吸收和选择吸收的特性。吸收分光光度法利用物质分子对电磁辐射的选择吸收特性,用于测量物质的吸收光谱,从而进行分析和研究。
					\item \textbf{朗伯定律}:\\
						朗伯定律(Lambert's law)是描述光线透过吸收性均匀介质时光强衰减规律的基本定律。朗伯定律的数学表示式为:
						\[ I = I_0 e^{-kl} \]
						吸收系数$k$是波长的函数,在一般吸收的波段内, $k$ 值很小,并且近乎于一常数;在选择吸收波段内, $k$ 值甚大,并且随波长的不同而有显著的变化。
						
						\begin{figure}[htbp]
							\centering
							\includegraphics[width=0.2\textwidth]{均匀介质.png}
							\caption{均匀媒质对光的吸收}
							\label{fig:graph6}
						\end{figure}
						
					\item \textbf{比尔定律}:\\
					比尔定律(Beer's law),也称为比尔-朗伯定律(Beer-Lambert law),是描述光线透过吸收性均匀介质时光强衰减规律的定律,是朗伯定律的一个特例。比尔定律的数学形式为:
					\[ A = \alpha c l \]
					$A$表示吸光度,$\alpha$是摩尔吸光系数,$c$为浓度,$l$是光通过溶液的路径长度。
					在比尔定律成立时,就可用测量吸收的方法来测定物质的浓度。这就是快速测定物质浓度的吸收光谱分析法。
						
						\begin{figure}[H]
							\centering
							\includegraphics[width=0.6\textwidth]{比尔图像.png}
							\caption{比尔定律示意图和吸收度、投射比标度的比较}
							\label{fig:graph7}
						\end{figure}
	%				\item 光谱仪和光学多通道分析仪:
						
					
				\end{enumerate}
			
		\end{enumerate}
	
	
	\subsection{实验前思考题}
		%思考题1
		\begin{question}
			日常生活中,光源可以分为热光源和冷光源,请分别说明太阳光、蜡烛、白炽灯、荧光灯、 LED 灯等属于哪一类光源,为什么?
		\end{question}
		
		在日常生活中,光源通常可以分为热光源和冷光源两类,具体如下:

\begin{enumerate}[label=\arabic*.]
    \item \textbf{热光源}:热光源是指其发光是由高温物质的热辐射产生的光线。这种光源通常是通过加热固体、液体或气体至非常高的温度来产生的。热光源的光谱通常是连续的,包含了各种波长的光线。太阳是典型的热光源,因为它的光是由太阳表面高温引起的热辐射所产生的。
    
    \item \textbf{冷光源}:冷光源是指其发光不是由高温物质的热辐射产生的,而是通过其他方式产生的光线,例如电击激发或化学反应等。冷光源的光谱通常是不连续的,具有明显的发射线。常见的冷光源包括荧光灯和 LED 灯等。
\end{enumerate}
综上所述,太阳光、蜡烛和白炽灯属于热光源,而荧光灯和 LED 灯属于冷光源。
	
	
	% 实验记录	
	\clearpage
	
	% 顶栏
	\begin{table}
		\renewcommand\arraystretch{1.7}
		\centering
		\begin{tabularx}{\textwidth}{|X|X|X|X|}
			\hline
			专业: & 物理学 & 年级: & 2022级 \\
			\hline
			姓名: &  黄罗琳& 学号: & 22344001\\
			\hline
			室温: &26℃  & 实验地点: & A501 \\
			\hline
			学生签名:&  \includegraphics[width=1cm]{签字.jpg} & 评分: &\\
			\hline
			实验时间:& 2024/4/18 & 教师签名:&\\
			\hline
		\end{tabularx}
	\end{table}
	% ---
	
	% 小标题
	\section{CA3 \quad 原子的发射和吸收光谱观测分析实验 \quad\heiti 实验记录}
	% ---
	
	% 实验过程记录
	\subsection{实验内容、步骤与结果}
	
	%
	\subsubsection{原子发射光谱的观测}
	\begin{enumerate}
		\item 拍摄钠灯光谱
		\begin{enumerate}
			\item 选择合适的积分次数和积分时间。实际实验中选择了“积分时间150ms,积分次数50次”
			
			\item 将光纤一端连接光谱仪,一端关闭,测量暗噪声。 
			
			\item 将光纤另一端装上支架,将支架对准未开启NA灯的环境,测量环境噪声。
			
			\item 选用S-d模式,将暗噪声与环境噪声都扣除后,将支架对准钠灯,测量钠灯光谱。
			
		\end{enumerate}
        软件给出如下实验图像,放大后经过寻峰后得出钠双黄线\\
		测得波长值分别为$\lambda_1=588.72nm$和$\lambda_2=589.34nm$
		\begin{figure}[!htb]
			\centering
			\begin{minipage}{0.35\textwidth}
				\centering
				\includegraphics[width=\linewidth]{钠.JPG}
				
				\label{fig:sodium_spectrum}
			\end{minipage}%
			\hspace{0.1\textwidth}
			\begin{minipage}{0.35\textwidth}
				\centering
				\includegraphics[width=\linewidth]{钠双黄线.JPG}
				
				\label{fig:sodium_double_yellow_lines}
			\end{minipage}
			\caption{钠的光谱图像}
			\label{fig:sodium_spectra}
		\end{figure}
		\item \textbf{拍摄汞灯光谱}
		重复实验步骤,测量汞灯光谱。
		\begin{figure}[H]
			\centering
			\begin{minipage}{0.35\textwidth}
				\centering
				\includegraphics[width=\linewidth]{汞灯.JPG}
			
				\label{fig:mercury_lamp1}
			\end{minipage}%
			\hspace{0.1\textwidth}
			\begin{minipage}{0.35\textwidth}
				\centering
				\includegraphics[width=\linewidth]{汞灯2.JPG}
				
				\label{fig:mercury_lamp2}
			\end{minipage}
		 \caption{汞灯的图像2}
			\label{fig:mercury_lamps}
		\end{figure}
		根据实验图像,可得545.90nm,435.59nm,404.41nm,578.85nm和576.65nm的谱线,最高峰是545.90nm
		\item \textbf{拍摄手机屏幕光谱}
		测量暗噪声和环境噪声后,将暗噪声与环境噪声都扣除,将支架对准手机屏幕,测量手机屏幕光谱。其中手机显示画面。不同颜色的纯色画面。
		\begin{figure}[{H}]
			\centering
			\includegraphics[width=0.8\linewidth]{蓝色.JPG}
			\caption{蓝色手机屏幕光谱}
			\label{}
		\end{figure}
		\begin{figure}[{H}]
			\centering
			\includegraphics[width=0.8\linewidth]{黄.JPG}
			\caption{黄色手机屏幕光谱}
			\label{}
		\end{figure}
				
	\end{enumerate}	

	\subsubsection{原子吸收光谱的观测}
	\begin{enumerate}
		\item \textbf{验证比尔定律}
			
			\begin{enumerate}
				\item 实验中所用溶液为清水和$KMnO_4$溶液,浓度为$0.1g/L$,$0.08g/L$,$0.06g/L$,$0.04g/L$,$0.02g/L$,$0.01g/L$.
				
				\item 按照实验一的做法,设置积分时间为50ms和积分次数20次,消除暗噪声和环境噪声。
				
				\item 选择吸光度模式,测量各种浓度$KMnO_4$溶液和清水的吸收曲线。
				
				\item 测量完所有浓度溶液的吸收曲线后,将所有数据放置在同一张图上进行对比。
				
			\end{enumerate}
			\begin{figure}[{H}]
				\centering
				\includegraphics[width=0.7\linewidth]{0.01.JPG}
				\caption{浓度为$0.01g/L$}
				\label{}
			\end{figure}
			
			\begin{figure}[H]
				\centering
				\includegraphics[width=0.7\linewidth]{0.02.JPG}
				\caption{浓度为$0.02g/L$}
				\label{fig:conc_0_02}
			\end{figure}
			
			\begin{figure}[H]
				\centering
				\includegraphics[width=0.7\linewidth]{0.04.JPG}
				\caption{浓度为$0.04g/L$}
				\label{fig:conc_0_04}
			\end{figure}
			
			\begin{figure}[H]
				\centering
				\includegraphics[width=0.7\linewidth]{0.06.JPG}
				\caption{浓度为$0.06g/L$}
				\label{fig:conc_0_06}
			\end{figure}
			
			\begin{figure}[H]
				\centering
				\includegraphics[width=0.7\linewidth]{0.08.JPG}
				\caption{浓度为$0.08g/L$}
				\label{fig:conc_0_08}
			\end{figure}
			
			\begin{figure}[H]
				\centering
				\includegraphics[width=0.7\linewidth]{0.1.JPG}
				\caption{浓度为$0.10g/L$}
				\label{fig:conc_0_10}
			\end{figure}
			\begin{figure}[H]
				\centering
				\includegraphics[width=0.7\linewidth]{总的.JPG}
				\caption{同一张图上显示}
				\label{fig:conc_0_10}
			\end{figure}
			\item \textbf{验证朗伯定律}
			
			\begin{enumerate}
				\item 按照实验一的做法,设置的积分时间50ms和积分次数20,消除暗噪声和环境噪声。
				
				\item 选择透射率模式,测量放置不同数量玻璃片的透射曲线。
				
				\item 测量完所有数量玻璃片的透射曲线后,将所有数据放置在同一张图上进行对比。
				
			\end{enumerate}
			\begin{figure}[{H}]
				\centering
				\includegraphics[width=0.8\linewidth]{总.JPG}
				\caption{不同数量玻璃片的透射率曲线}
				\label{}
			\end{figure}

	\end{enumerate}
	
	% ---
	
	% 原始数据


	% ---
	
	% 问题记录
	\subsection{实验过程遇到问题及解决办法}
	\begin{enumerate}
		\item 实验中出现的实验图像存在较大的噪声,例如在不同浓度$KMnO_4$溶液的实验中参考光谱的曲线可以明显看到噪声的存在,并且这也体现在后续的实验图像中,例如图13的左侧和图14的并不平滑的曲线。
		\begin{figure}[{H}]
			\centering
			\includegraphics[width=0.8\linewidth]{背景清水.JPG}
			\caption{参考光谱曲线}
			\label{}
		\end{figure}
		\item 实验中要时刻要注意测量值不能饱和,否则各个曲线之间基本毫无差异。
		\item 注意在$KMnO_4$溶液的实验中,要注意实验仪器的标签不能阻挡实验光路,否则实验结果会出现严重偏差。
	\end{enumerate}
	% ---
	
	
	
	% 分析与讨论	
	\clearpage
	
	% 顶栏
	\begin{table}
		\renewcommand\arraystretch{1.7}
		\begin{tabularx}{\textwidth}{|X|X|X|X|}
			\hline
			专业:& 物理学 &年级:& 2022级\\
			\hline
			姓名: & 黄罗琳 & 学号:& 22344001\\
			\hline
			日期:& 2024/4/18 & 评分: &\\
			\hline
		\end{tabularx}
	\end{table}
	% ---
	
	% 小标题
	\section{CA3 \quad 原子的发射和吸收光谱观测分析实验 \quad\heiti 分析与讨论}
	% ---
	
	% 数据处理
	\subsection{实验数据分析}
	\subsubsection{实验一 \quad 原子发射光谱的观测}
	\begin{enumerate}
		\item \textbf{拍摄钠灯光谱}
		
		
			
			
			钠原子光谱的主线系只有$\widetilde{v}=3p \rightarrow 3s$ 的二条谱线(钠双黄线)是在可见区,其余在紫外区。
			则根据公式:
			\[
			\widetilde{v}=R(\frac{1}{n_2^{*2}}-\frac{1}{n_1^{*2}})=\frac{R}{{(n'-{\mu'}_{l'})}^2}-\frac{R}{{(n-\mu_l)}^2}
			\]
			
			初态为$3p$能级,$n=3$,$l=1$,$\mu_l=0.883$;末态为$3s$能级,$n'=3$,$l'=0$,$\mu'_{l'}=1.373$
			
			则代入公式,计算得到
			\[ \frac{1}{\lambda}\approx\frac{1}{589.3nm}
			\]
			
			至于双黄线的具体波长数值,一般是通过实验进行测量,其非常稳定,一般作为仪器的校准和实验模型的校对。

			\begin{table}[H]
				\centering
				\caption{实验值与理论值的相对误差}
				\begin{tabular}{|c|c|c|c|}
				\hline
				对比组 & 实验值 (nm) & 理论值 (nm) & 相对误差 (\%) \\
				\hline
				1 & 589.28 & 589.0 & 0.475 \\
				2 & 589.91 & 589.6 & 0.052 \\
				\hline
				\end{tabular}
				\end{table}
				\begin{figure}[{H}]
					\centering
					\includegraphics[width=0.5\linewidth]{钠双黄线.JPG}
					\caption{钠双黄线}
					\label{}
				\end{figure}
				观察图16可以发现钠双黄线存在约0.3nm的线宽,对于此有两种猜想:
				\begin{enumerate}
					\item 多普勒致宽 \quad 多普勒致宽是由于发射光的原子在热运动中产生的。由于某些原子向观测者移动,而另一些原子远离观测者,这会导致频率的蓝移和红移。其结果是光谱线的增宽。多普勒效应与绝对温度的平方根成正比,因此,温度越高,速度分布越广,致宽越显著。
					\item 压力致宽 \quad 压力致宽主要与原子或分子的相互作用有关。在高压环境中,原子之间的碰撞或相互作用会干扰发光过程,造成谱线的增宽。压力致宽与原子的数密度有关,数密度越高,碰撞几率越大,导致谱线致宽。因此,它与压强成正比。
					



					\end{enumerate}
					这些可能导致了存在一定的实验误差和0.3nm的线宽。
				

		
			
		
		\item \textbf{拍摄汞灯光谱}
			
			
		\begin{figure}[{H}]
			\centering
			\includegraphics[width=0.8\linewidth]{汞灯.JPG}
			\caption{汞灯谱线}
			\label{}
		\end{figure}
		
			\begin{table}[H]
				\centering
				\begin{tblr}{
					cells = {c},
					vline{1-2,5} = {-}{},
					hline{1-2,7} = {-}{},
				}
				谱线颜色 & 实验值/nm & 标准值/nm & 相对误差   \\
				黄色   & 579.27 & 579.1  & 0.029\% \\
				黄色   & 577.23 & 577.0  & 0.040\% \\
				绿色   & 546.35 & 546.1  & 0.046\% \\
				蓝色   & 436.15 & 435.8  & 0.080\% \\
				紫色   & 404.99 & 404.7  & 0.072\% 
				\end{tblr}
				\caption{理论值与实验值对比}
				\label{tab:table2}
			\end{table}
			实验相对误差均较小,符合实验较为成功。
			\item \textbf{拍摄手机屏幕光谱}\\
			手机屏幕通常使用发光二极管(LED)技术来显示图像。不同的LED基于其所使用的半导体材料会发出特定颜色的光。在手机屏幕中,红色的光通常是由含铝的镓砷化合物(如铝镓砷,AlGaAs)或镓磷化合物(如镓磷化铝,AlInGaP)生成的;绿色的光通常来自铟镓磷(InGaN);而蓝色的光由氮化镓(GaN)生成。这些不同的LED材料和结构决定了它们在光谱中的表现,分别对应红、绿、蓝三种颜色。

			LED的发光原理基于电子的能级跃迁。当电流流过半导体材料时,电子会被激发到更高的能级,随后再返回到较低的能级。在这一过程中,电子会释放出一定能量的光子,产生光。不同的半导体材料拥有不同的能级结构,这意味着它们产生的光子的能量和频率各不相同。正因为如此,在手机屏幕中,红色、绿色和蓝色LED所发出的光会形成一个可见光谱的连续范围,从而展现出丰富的色彩。
			
			所以黄色的手机屏幕有红色和绿色的LED进行发出,得出如图所示的光谱。
			\begin{figure}[{H}]
				\centering
				\includegraphics[width=0.4\linewidth]{黄.JPG}
				\caption{黄色手机屏幕光谱}
				\label{}
			\end{figure}
			
	\end{enumerate}
			\subsubsection{实验二 原子吸收光谱的观测}
			
		\begin{enumerate}
			\item \textbf{验证比尔定律}
			
			由于实验数据的原图噪声较大,所以通过Python代码滤波之后并进行寻峰操作,代码见实验报告附录。
			\begin{figure}[{H}]
				\centering
				\includegraphics[width=0.8\linewidth]{总的.JPG}
				\caption{未滤波前实验数据图像}
				\label{}
			\end{figure}
			\begin{figure}[{H}]
				\centering
				\includegraphics[width=0.8\linewidth]{KMNO4滤波.png}
				\caption{不同浓度的$KMnO_4$溶液对吸收率的影响(三个吸收峰)}
			\end{figure}
			\begin{table}
				\centering
				\begin{tblr}{
						cells = {c},
						vline{2} = {-}{},
						hline{1-2,8} = {-}{},
					}
					浓度/(g/L) & 吸收峰1  & 吸收峰2  & 吸收峰3  \\
					0.1     & 1.43 & 1.73 & 1.66 \\
					0.08    & 1.22 & 1.45 & 1.36 \\
					0.06     & 0.83 & 0.96 & 0.89 \\
					0.04     & 0.63 & 0.73 & 0.66 \\
					0.02     & 0.47 & 0.47 & 0.42 \\
					0.01      & 0.21 & 0.22 & 0.21 
				\end{tblr}
				\caption{同一吸收峰对应的不同浓度溶液的吸收率}
			
			\end{table}
			\begin{figure}[{H}]
				\centering
				\includegraphics[width=0.6\linewidth]{线性拟合.png}
				\caption{线性拟合图像}
				\label{}
			\end{figure}
			则以浓度为横轴,吸收率为纵轴,绘制线性拟合图,最终绘制结果如图21所示。
			比尔定律的数学形式为:
				\[ A = \alpha c l \]
				$A$表示吸光度,$\alpha$是摩尔吸光系数,$c$为浓度,$l$是光通过溶液的路径长度,其中前两个为固定系数,所以为线性关系。即只需要验证$A$与$c$的线性关系,即可验证比尔定律。
				
				由图21可知,三条相关系数均在0.98以上,所以实验数据存在线性关系,即比尔定律成立。
				\item \textbf{验证朗伯定律}
				
				由于实验数据噪声较大,进行不同程度的滤波(参数越小滤波效果越大)
				\begin{figure}[{H}]
					\centering
					\includegraphics[width=0.7\linewidth]{1阶滤波.png}
					\caption{滤波效果(参数为1)}
					\label{}
				\end{figure}
				\begin{figure}[{H}]
					\centering
					\includegraphics[width=0.7\linewidth]{2阶滤波.png}
					\caption{滤波效果(参数为2)}
					\label{}
				\end{figure}
				提取实验数据中500nm,550nm,600nm,650nm,700nm波长分析,玻璃片数量做横轴,透射率的自然对数做纵轴,绘制线性拟合图像。
			\begin{figure}[{H}]
				\centering
				\includegraphics[width=0.8\linewidth]{玻璃拟合.png}
				\caption{线性拟合图像}
				\label{}
			\end{figure}
			朗伯定律的数学公式为:

\[ I = I_0 \cdot e^{-kl} \]

其中,\( I_0 \) 是入射光的强度,\( I \) 是透射光的强度,\( k \) 是吸收系数,\( l \) 是材料的厚度。

对公式进行对数变换,可以得到:

\[ \ln \frac{I}{I_0} = -kl \]

这个公式说明了透射率的自然对数与材料厚度之间的关系。如果这两个变量之间存在线性关系,朗伯定律则成立。

为了验证这一点,可以测量不同厚度材料的透射率 \( T = \frac{I}{I_0} \),然后计算透射率的自然对数:

\[ \ln T = \ln \frac{I}{I_0} \]

将这个值与材料厚度 \( l \) 进行线性回归。

根据变形后的公式,这种线性关系的斜率是吸收系数 \( k \) 的相反数。而根据图24所绘制的图像R=-1,说明存在很强的负相关关系,即透射率的自然对数与材料厚度之间有强烈的线性关系,从而验证了朗伯定律。










	\end{enumerate}
	
	
	% 实验后思考题
	\subsection{实验后思考题}
	
	%思考题1
	\begin{question}
		钠灯光谱有哪些特征?能否从光谱图上判别各谱线所属线系?举例说明。想·
	\end{question}
	钠灯的光谱具有以下特征:

\begin{enumerate}
    \item \textbf{双线结构}:钠灯的光谱由两条主要的谱线组成,被称为钠D线,分别位于589.0 nm和589.6 nm。这两条谱线在光谱图上通常很容易辨认。

    \item \textbf{窄线宽}:钠灯的谱线非常窄,说明其光源具有较高的单色性。窄线宽通常意味着谱线较为纯净。

    \item \textbf{强度不均}:钠D线中的两条谱线强度通常不相等,589.0 nm的D2线通常比589.6 nm的D1线稍强。这种强度不均可以从光谱图上观察到。
\end{enumerate}

钠原子的光谱包含四个主要线系:主线系、锐线系、漫线系和基线系。通过观察谱线的形状和宽度,可以判别各谱线所属的线系。

\begin{itemize}
    \item \textbf{主线系}:包含最强的谱线,如钠D线。谱线通常呈现尖锐且窄的峰值。

    \item \textbf{锐线系}:强度略低,但形状与主线系相似。一般呈现窄而尖锐的峰值,表明谱线纯净。

    \item \textbf{漫线系}:这些谱线通常较宽且较平滑,说明光谱可能受多种因素影响。

    \item \textbf{基线系}:最弱的谱线,呈现为淡淡的峰值,可能需要高灵敏度的光谱仪才能检测到。
\end{itemize}

要从光谱图上判断各个线系,可以观察谱线的形状和宽度。如果谱线尖锐且窄,可能属于锐线系;而宽而平滑的谱线则可能属于漫线系。

	% 思考题2
	\begin{question}
		在发射光谱和吸收光谱测量中,光路有何异同?
	\end{question}
	\textbf{相同点}:
	
		都是由光源发射光经过光纤进入光栅光谱仪,然后软件根据得到的数据结合相关去除噪声的算法进行数据绘图。
	
		
	
	
	\textbf{不同点}:
	
	在发射光谱的测量实验中,光源发出的光无其他元件的情况下,经过光纤导入进光谱仪;\\
		在吸收光谱的实验中,光源发出的光首先会经过样品,然后再通过光纤导入进光谱仪。
		
		
	
	
	% 思考题3
	\begin{question}
		根据高锰酸钾溶液的吸收光谱, 应如何选择理想光源,为什么?
	\end{question}
	根据实验所得的高锰酸钾溶液吸收光谱,在选择理想光源时需要考虑这些因素:
	\begin{enumerate}
		\item 波长范围:高锰酸钾溶液在水溶液中呈现紫色,其主要吸收光谱位于 500 nm 至 600 nm 范围内,吸收峰在 525 nm 附近,因此需要选择光源覆盖这一范围。
		\item 光源素质:具有稳定且均匀的光强,以确保光在经过高锰酸钾溶液后仍然强度足够高,以提高信噪比,并且光源的相关参数要稳定可调,保证实验的基本条件不变。
	\end{enumerate}
	故综合以上因素,常用光源为氙灯和钨灯,此两种均能够覆盖高锰酸钾溶液的吸收光谱的范围,并且具有较好的稳定性。此外可以选用特定波长LED光源,可以精确测量某个波长的实验数据。
	% ---
	
	
	% 结语部分
	\clearpage
	
	% 小标题
	\section{CA3 \quad 原子的发射和吸收光谱观测分析实验 \quad\heiti 结语}
	% ---
	\subsection{实验所用代码}
\begin{figure}[H]
	\begin{minipage}[b]{0.3\linewidth}
	  \centering
	  \includegraphics[width=0.7\linewidth]{拟合一代码.png}
	  \caption{线性拟合1(图21)}
	\end{minipage}
	\hfill
	\begin{minipage}[b]{0.3\linewidth}
	  \centering
	  \includegraphics[width=0.7\linewidth]{拟合2.png}
	  \caption{线性拟合2(图24)}
	\end{minipage}
	\hfill
	\begin{minipage}[b]{0.3\linewidth}
	  \centering
	  \includegraphics[width=0.7\linewidth]{寻峰代码.png}
	  \caption{滤波寻峰代码(图20)}
	\end{minipage}
\end{figure}
	% 总结、杂谈与致谢
	\subsection{实验心得和体会、意见建议等}
	\begin{enumerate}
		\item 实验总体难度不难,但是实验可能会存在仪器问题,导致噪声较大,本次实验中便遇到了相关问题,通过后续Python代码的编辑滤波寻峰等操作,完成了实验数据的净化。
		\item 实验中出现了由于实验仪器本身的颜色引入了噪声,可以通过改变实验的初始设置条件来进行系统误差的去除。
		\item \textbf{本实验报告采用LATEX编辑,实验所有要求内容均包含于本实验报告中}
	\end{enumerate}
	\quad \large \textbf{感谢您对于此篇实验报告的阅读与批改,祝您工作顺利!}
	% ---
	

	% 附件
	\subsection{附件}
	\begin{figure}[H]
		\begin{minipage}[b]{0.3\linewidth}
		  \centering
		  \includegraphics[width=0.5\linewidth]{原始数据.jpg}
		  \caption{实验记录}
		\end{minipage}
		\hfill
		\begin{minipage}[b]{0.3\linewidth}
		  \centering
		  \includegraphics[width=0.5\linewidth]{桌面.jpg}
		  \caption{桌面整理}
		\end{minipage}
		\hfill
		\begin{minipage}[b]{0.3\linewidth}
		  \centering
		  \includegraphics[width=0.5\linewidth]{预习.jpg}
		  \caption{预习报告签字}
		\end{minipage}
	\end{figure}

	% ---
	
	
\end{document}